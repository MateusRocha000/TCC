\chapter[Referencial Teórico]{Referencial Teórico}
\label{cap:referencial}

Neste capítulo serão discutidos os conceitos fundamentais para a compreensão deste trabalho. Primeiramente, serão definidos os conceitos das linguagens utilizadas na programação Web, HTML, CSS e JavaScript. Também será explicado o conceito de gamificação, apresentado no capítulo anteior.

\section{HTML, CSS e JavaScript}
\subsection{HTML}
O HTML, acrônimo de Hypertext Markup Language, ou em português Linguagem de Marcação de Hipertexto, é a linguagem base da Internet. Ela consiste em utilizar várias tags, marcações, para definir os elementos da página Web. Assim, o navegador saberá o que cada elemento significa, seja um título, um parágrafo ou uma imagem. \cite{html} Um documento HTML tem como estrutura básica a apresentada à seguir:

\begin{verbatim}
<!DOCTYPE html>

<html lang="pt-br">
<head>
    <meta charset="utf-8">
    <title>Título da página</title>
</head>
<body>
   <h1>Aqui vai o texto do título</h1>
   <p>Aqui vai o texto do parágrafo.</p>
</body>
</html>
\end{verbatim}

Um arquivo HTML precisa, inicialmente, ser aberto com a tag <html> com "!DOCTYPE" antes para que o navegador identifique o tipo de documento que ele está carregando. A tag html serve para informar que tudo que estiver dentro dela é escrito em HTML.

Em seguida, temos a tag <head> onde está indicado o título do documento. Este título, indicado pela tag <title>, é utilizado pelos sistemas de busca para mostrar o título da página.

Finalmente, temos a tag <body>. Dentro desta estará o conteúdo da sua página. Tudo que deverá ser exibido na página deve estar dentro dessa tag. As tags utilizadas neste trabalho serviram para a inclusão de botões, imagens, parágrafos, cada um com sua respectiva tag. 

\subsection{CSS}
CSS é a sigla para o termo em inglês Cascading Style Sheets, ou em português Folha de Estilo em Cascatas. Seu papel é dar estilo à página, ajustando, por exemplo, a cor de fundo, a posição do texto na página, a fonte ou, até mesmo, criação de tabelas.

Para que um elemento seja modificado, são usados os seletores e as declarações. Os seletores são os elementos selecionados para receberem uma propriedade. Ou seja, para que um parágrafo seja modificado, é necessário utilizar sua tag \textit{p} para utilizar as propriedades.


\subsection{JavaScript}

\section{Gamificação}