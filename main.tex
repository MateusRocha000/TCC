%% Baseado no arquivo: 
%% abtex2-modelo-trabalho-academico.tex, v-1.9.6 laurocesar
%% by abnTeX2 group at http://www.abntex.net.br/ 
%% Adaptado para um modelo dssse TCC (Graduação)

% ------------------------------------------------------------------------
% ------------------------------------------------------------------------
% abnTeX2: Modelo de Trabalho Academico (tese de doutorado, dissertacao de
% mestrado e trabalhos monograficos em geral) em conformidade com 
% ABNT NBR 14724:2011: Informacao e documentacao - Trabalhos academicos -
% Apresentacao
% ------------------------------------------------------------------------
% ------------------------------------------------------------------------

\documentclass[
	% -- opções da classe memoir --
	12pt,				% tamanho da fonte
	openright,			% capítulos começam em pág ímpar (insere página vazia caso preciso)
	twoside,			% para impressão em recto e verso. Oposto a oneside
	a4paper,			% tamanho do papel. 
	% -- opções da classe abntex2 --
	%chapter=TITLE,		% títulos de capítulos convertidos em letras maiúsculas
	%section=TITLE,		% títulos de seções convertidos em letras maiúsculas
	%subsection=TITLE,	% títulos de subseções convertidos em letras maiúsculas
	%subsubsection=TITLE,% títulos de subsubseções convertidos em letras maiúsculas
	% -- opções do pacote babel --
	english,			% idioma adicional para hifenização
	%french,				% idioma adicional para hifenização
	%spanish,			% idioma adicional para hifenização
	brazil				% o último idioma é o principal do documento
	]{abntex2}

% ---
% Pacotes básicos 
% ---
\usepackage{lmodern}			% Usa a fonte Latin Modern			
\usepackage[T1]{fontenc}		% Selecao de codigos de fonte.
\usepackage[utf8]{inputenc}		% Codificacao do documento (conversão automática dos acentos)
\usepackage{lastpage}			% Usado pela Ficha catalográfica
\usepackage{indentfirst}		% Indenta o primeiro parágrafo de cada seção.
\usepackage{color}				% Controle das cores
\usepackage{graphicx}			% Inclusão de gráficos
\usepackage{microtype} 			% para melhorias de justificação
% ---
\usepackage{listings}

\definecolor{dkgreen}{rgb}{0,0.6,0}
\definecolor{gray}{rgb}{0.5,0.5,0.5}
\definecolor{mauve}{rgb}{0.58,0,0.82}

\lstset{language=Html}


% ---
% Pacotes de citações
% ---
\usepackage[brazilian,hyperpageref]{backref}	 % Paginas com as citações na bibl
\usepackage[alf]{abntex2cite}	% Citações padrão ABNT

% --- 
% CONFIGURAÇÕES DE PACOTES
% --- 

% ---
% Configurações do pacote backref
% Usado sem a opção hyperpageref de backref
\renewcommand{\backrefpagesname}{%Citado na(s) página(s):~
}
% Texto padrão antes do número das páginas
\renewcommand{\backref}{}
% Define os textos da citação
\renewcommand*{\backrefalt}[4]{
	%\ifcase #1 %
	%	Nenhuma citação no texto.%
	%\or
	%	Citado na página #2.%
	%\else
	%	Citado #1 vezes nas páginas #2.%
	%\fi
    }%
% ---

% ---
% Informações de dados para CAPA e FOLHA DE ROSTO
% ---
\titulo{Desenvolvimento de plataforma online para ensino de Web com gamificação}
\autor{Mateus Rocha Gonçalves}
\local{Belo Horizonte}
\data{2018}
\orientador{Daniel Hasan Dalip}
\instituicao{%
  Centro Federal de Educação Tecnológica de Minas Gerais -- CEFET-MG
  \par
  Departamento de Computação
  \par
  Curso de Engenharia da Computação
  }
\tipotrabalho{Monografia (Graduação)}
% O preambulo deve conter o tipo do trabalho, o objetivo, 
% o nome da instituição e a área de concentração 
\preambulo{Trabalho de Conclusão de Curso apresentado ao Curso
de Engenharia de Computação do Centro Federal de
Educação Tecnológica de Minas Gerais.}
% ---


% ---
% Configurações de aparência do PDF final

% alterando o aspecto da cor azul
\definecolor{blue}{RGB}{41,5,195}

% informações do PDF
\makeatletter
\hypersetup{
     	%pagebackref=true,
		pdftitle={\@title}, 
		pdfauthor={\@author},
    	pdfsubject={\imprimirpreambulo},
	    pdfcreator={LaTeX with abnTeX2},
		pdfkeywords={abnt}{latex}{abntex}{abntex2}{trabalho acadêmico}, 
		colorlinks=true,       		% false: boxed links; true: colored links
    	linkcolor=blue,          	% color of internal links
    	citecolor=blue,        		% color of links to bibliography
    	filecolor=magenta,      		% color of file links
		urlcolor=blue,
		bookmarksdepth=4
}
\makeatother
% --- 

% --- 
% Espaçamentos entre linhas e parágrafos 
% --- 

% O tamanho do parágrafo é dado por:
\setlength{\parindent}{1.3cm}

% Controle do espaçamento entre um parágrafo e outro:
\setlength{\parskip}{0.2cm}  % tente também \onelineskip

% ---
% compila o indice
% ---
\makeindex
% ---

% ----
% Início do documento
% ----
\begin{document}

% Seleciona o idioma do documento (conforme pacotes do babel)
%\selectlanguage{english}
%\selectlanguage{brazil}

% Retira espaço extra obsoleto entre as frases.
\frenchspacing 

% ----------------------------------------------------------
% ELEMENTOS PRÉ-TEXTUAIS
% ----------------------------------------------------------



%% Baseado no arquivo: 
%% abtex2-modelo-trabalho-academico.tex, v-1.9.6 laurocesar
%% by abnTeX2 group at http://www.abntex.net.br/ 
%% Adaptado para um modelo de TCC (Graduação)

% ---
% Capa
% ---
\imprimircapa
% ---

% ---
% Folha de rosto
% (o * indica que haverá a ficha bibliográfica)
% ---
\imprimirfolhaderosto*
% ---

% ---
% Inserir a ficha bibliografica
% ---

% Isto é um exemplo de Ficha Catalográfica, ou ``Dados internacionais de
% catalogação-na-publicação''. Você pode utilizar este modelo como referência. 
% Porém, provavelmente a biblioteca da sua universidade lhe fornecerá um PDF
% com a ficha catalográfica definitiva após a defesa do trabalho. Quando estiver
% com o documento, salve-o como PDF no diretório do seu projeto e substitua todo
% o conteúdo de implementação deste arquivo pelo comando abaixo:
%
% \begin{fichacatalografica}
%     
% \end{fichacatalografica}

\begin{fichacatalografica}
~
%\includepdf{fig_ficha_catalografica.pdf}
% 	\sffamily
% 	\vspace*{\fill}					% Posição vertical
% 	\begin{center}					% Minipage Centralizado
% 	\fbox{\begin{minipage}[c][8cm]{13.5cm}		% Largura
% 	\small
% 	\imprimirautor
% 	%Sobrenome, Nome do autor
	
% 	\hspace{0.5cm} \imprimirtitulo  / \imprimirautor. --
% 	\imprimirlocal, \imprimirdata-
	
% 	\hspace{0.5cm} \pageref{LastPage} p. : il. (algumas color.) ; 30 cm.\\
	
% 	\hspace{0.5cm} \imprimirorientadorRotulo~\imprimirorientador\\
	
% 	\hspace{0.5cm}
% 	\parbox[t]{\textwidth}{\imprimirtipotrabalho~--~\imprimirinstituicao,
% 	\imprimirdata.}\\
	
% 	\hspace{0.5cm}
% 		1. Palavra-chave1.
% 		2. Palavra-chave2.
% 		2. Palavra-chave3.
% 		I. Orientador.
% 		II. Universidade xxx.
% 		III. Faculdade de xxx.
% 		IV. Título 			
% 	\end{minipage}}
% 	\end{center}
\end{fichacatalografica}
% ---

% % ---
% % Inserir errata
% % ---
% \begin{errata}
% Elemento opcional da \citeonline[4.2.1.2]{NBR14724:2011}. Exemplo:

% \vspace{\onelineskip}

% FERRIGNO, C. R. A. \textbf{Tratamento de neoplasias ósseas apendiculares com
% reimplantação de enxerto ósseo autólogo autoclavado associado ao plasma
% rico em plaquetas}: estudo crítico na cirurgia de preservação de membro em
% cães. 2011. 128 f. Tese (Livre-Docência) - Faculdade de Medicina Veterinária e
% Zootecnia, Universidade de São Paulo, São Paulo, 2011.

% \begin{table}[htb]
% \center
% \footnotesize
% \begin{tabular}{|p{1.4cm}|p{1cm}|p{3cm}|p{3cm}|}
%   \hline
%    \textbf{Folha} & \textbf{Linha}  & \textbf{Onde se lê}  & \textbf{Leia-se}  \\
%     \hline
%     1 & 10 & auto-conclavo & autoconclavo\\
%    \hline
% \end{tabular}
% \end{table}

% \end{errata}
% ---



%

% ---

% ---

% ---
% RESUMOS
% ---

% resumo em português
\setlength{\absparsep}{18pt} % ajusta o espaçamento dos parágrafos do resumo
\begin{resumo}
Este é o resumo de seu Trabalho de Conclusão de Curso (TCC). 
Nele, é importante apresentar de forma sucinta o problema a ser tratado e sua contextualização, 
a motivação e o objetivo de seu trabalho.
Logo após, apresente os resultados obtidos, contribuições e conclusões obtidas.

 \textbf{Palavras-chave}: Programação. Web. HTML. CSS. JavaScript. Gamificação.
\end{resumo}

% resumo em inglês
\begin{resumo}[Abstract]
 \begin{otherlanguage*}{english}
   This is the abstract of your undergraduate thesis. 
   Here is important to contextualize and present your problem, motivation and the goal of your work. After that, you need to present the results, contributions and conclusions of your study.

   \vspace{\onelineskip}
 
   \noindent 
   \textbf{Keywords}: Programming. Web. HTML. CSS. JavaScript. Gamification.
 \end{otherlanguage*}
\end{resumo}


% ---

% ---
% inserir lista de ilustrações
% ---
\pdfbookmark[0]{\listfigurename}{lof}
\listoffigures*
\cleardoublepage
% ---

% ---
% inserir lista de tabelas
% ---
\pdfbookmark[0]{\listtablename}{lot}
\listoftables*
\cleardoublepage
% ---

% ---
% inserir lista de abreviaturas e siglas
% ---
\begin{siglas}
  \item[ABNT] Associação Brasileira de Normas Técnicas
  \item[abnTeX] ABsurdas Normas para TeX
\end{siglas}
% ---

% ---
% inserir lista de símbolos
% ---
\begin{simbolos}
  \item[$ \Gamma $] Letra grega Gama
  \item[$ \Lambda $] Lambda
  \item[$ \zeta $] Letra grega minúscula zeta
  \item[$ \in $] Pertence
\end{simbolos}
% ---

% ---
% inserir o sumario
% ---
\pdfbookmark[0]{\contentsname}{toc}
\tableofcontents*
\cleardoublepage
% ---

% ----------------------------------------------------------
% ELEMENTOS TEXTUAIS
% ----------------------------------------------------------
\textual

% ----------------------------------------------------------
% Como o documento será grande, sugiro dividir em diversos arquivos, um para cada capítulo.
% ----------------------------------------------------------


\chapter[Introdução]{Introdução}
\label{cap:introducao}

A Internet teve seu início em 1969 com o nome de Arpanet nos EUA. Seus criadores eram pesquisadores que precisam de criar uma forma de comunicar os laboratórios de pesquisa em tempo de Guerra Fria, garantindo a troca de informação entre militares e cientistas. \cite{folha} 

Nas décadas de 1970 e 1980, a Internet começou a ser utilizada no meio acadêmico, onde os estudantes e os professores se comunicavam através de mensagens. Em 1990, o alcance da Internet aumentou e foi desenvolvida a World Wide Web, por Tim Bernes-Lee, dinamizando a criação de sites e, então, dando a oportunidade para o surgimento de navegadores, como Internet Explorer. Através desses, usuários buscavam informações sobre assuntos escolares, oportunidades de emprego, jogos, entre vários outros. Além disso, empresas viram a Internet como uma incrível forma de aumentar seus lucros transformando-a em shopping centers virtuais. \cite{internet} E para este e muitos outros usos, há a necessidade de se criar um portal Web para disponibilizar as informações.

Tendo em vista essa necessidade de um portal na Internet para disponibilizar informações, este trabalho tem como proposta o ensino dos conceitos de Web design, como HTML, CSS e JavaScript, para as pessoas através de uma plataforma online aliado ao conceito de gamificação, que será explicado posteriormente. 

Para que o objetivo do trabalho seja alcançado, haverão etapas a serem seguidas. Primeiramente, será feita uma revisão das funções da programação Web, para que seja definido o que o usuário irá aprender. Posteriormente, será desenvolvida uma plataforma online que ensine cada elemento da programação ao mesmo tempo que o usuário resolva problemas utilizando o que aprendeu. Por fim, serão coletadas as opiniões dos usuários sobre o conteúdo ensinado e sobre a plataforma para que a avaliação acerca da mesma seja realizada.

Ao final deste projeto, espera-se que os usuários adquiram um interesse maior sobre a área de programação Web, tendo a plataforma como uma ideia inicial do que pode ser criado com essa programação. Além disso, como um dos principais resultados, espera-se que eles aprendam, de maneira sucinta, os conceitos ensinados. Este trabalho contribui para a demonstração de como a gamificação pode ser utilizada para o ensino nas escolas como uma forma de atrair a atenção dos alunos para as aulas.

% ----------------------------------------------------------


\chapter[Referencial Teórico]{Referencial Teórico}
\label{cap:referencial}

Neste capítulo serão discutidos os conceitos fundamentais para a compreensão deste trabalho. Primeiramente, serão definidos os conceitos das linguagens utilizadas na programação Web, HTML, CSS e JavaScript. Também será explicado o conceito de gamificação, apresentado no capítulo anteior.

\section{HTML, CSS e JavaScript}
\subsection{HTML}
O HTML, acrônimo de Hypertext Markup Language, ou em português Linguagem de Marcação de Hipertexto, é a linguagem base da Internet. Ela consiste em utilizar várias tags, marcações, para definir os elementos da página Web. Assim, o navegador saberá o que cada elemento significa, seja um título, um parágrafo ou uma imagem. \cite{html} Um documento HTML tem como estrutura básica a apresentada à seguir:

\begin{verbatim}
<!DOCTYPE html>

<html lang="pt-br">
<head>
    <meta charset="utf-8">
    <title>Título da página</title>
</head>
<body>
   <h1>Aqui vai o texto do título</h1>
   <p>Aqui vai o texto do parágrafo.</p>
</body>
</html>
\end{verbatim}

Um arquivo HTML precisa, inicialmente, ser aberto com a tag <html> com "!DOCTYPE" antes para que o navegador identifique o tipo de documento que ele está carregando. A tag html serve para informar que tudo que estiver dentro dela é escrito em HTML.

Em seguida, temos a tag <head> onde está indicado o título do documento. Este título, indicado pela tag <title>, é utilizado pelos sistemas de busca para mostrar o título da página.

Finalmente, temos a tag <body>. Dentro desta estará o conteúdo da sua página. Tudo que deverá ser exibido na página deve estar dentro dessa tag. As tags utilizadas neste trabalho serviram para a inclusão de botões, imagens, parágrafos, cada um com sua respectiva tag. 

\subsection{CSS}
CSS é a sigla para o termo em inglês Cascading Style Sheets, ou em português Folha de Estilo em Cascatas. Seu papel é dar estilo à página, ajustando, por exemplo, a cor de fundo, a posição do texto na página, a fonte ou, até mesmo, criação de tabelas.

Para que um elemento seja modificado, são usados os seletores e as declarações. Os seletores são os elementos selecionados para receberem uma propriedade. Ou seja, para que um parágrafo seja modificado, é necessário utilizar sua tag \textit{p} para utilizar as propriedades.


\subsection{JavaScript}

\section{Gamificação}

\chapter[Trabalhos Relacionados]{Trabalhos Relacionados}
\label{cap:trab_relacionados}

\chapter[Metodologia]{Metodologia}
\label{cap:metodologia}

\chapter[Andamento do Trabalho]{Andamento do Trabalho}
\label{cap:andamento}

\chapter[Conclusão]{Conclusão}
\label{cap:conclusao}

% ----------------------------------------------------------
% Finaliza a parte no bookmark do PDF
% para que se inicie o bookmark na raiz
% e adiciona espaço de parte no Sumário
% ----------------------------------------------------------
\phantompart







% ----------------------------------------------------------
% ELEMENTOS PÓS-TEXTUAIS
% ----------------------------------------------------------

%% Baseado no arquivo: 
%% abtex2-modelo-trabalho-academico.tex, v-1.9.6 laurocesar
%% by abnTeX2 group at http://www.abntex.net.br/ 
%% Adaptado para um modelo de TCC (Graduação)

\postextual
% ----------------------------------------------------------

% ----------------------------------------------------------
% Referências bibliográficas
% ----------------------------------------------------------
\bibliography{referencias}





\end{document}
