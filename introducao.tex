\chapter[Introdução]{Introdução}
\label{cap:introducao}

A Internet teve seu início em 1969 com o nome de Arpanet nos EUA. Seus criadores eram pesquisadores que precisam de criar uma forma de comunicar os laboratórios de pesquisa em tempo de Guerra Fria, garantindo a troca de informação entre militares e cientistas. \cite{folha} 

Nas décadas de 1970 e 1980, a Internet começou a ser utilizada no meio acadêmico, onde os estudantes e os professores se comunicavam através de mensagens. Em 1990, o alcance da Internet aumentou e foi desenvolvida a World Wide Web, por Tim Bernes-Lee, dinamizando a criação de sites e, então, dando a oportunidade para o surgimento de navegadores, como Internet Explorer. Através desses, usuários buscavam informações sobre assuntos escolares, oportunidades de emprego, jogos, entre vários outros. Além disso, empresas viram a Internet como uma incrível forma de aumentar seus lucros transformando-a em shopping centers virtuais. \cite{internet} E para este e muitos outros usos, há a necessidade de se criar um portal Web para disponibilizar as informações.

Tendo em vista essa necessidade de um portal na Internet para disponibilizar informações, este trabalho tem como proposta o ensino dos conceitos de Web design, como HTML, CSS e JavaScript, para as pessoas através de uma plataforma online aliado ao conceito de gamificação, que será explicado posteriormente. 

Para que o objetivo do trabalho seja alcançado, haverão etapas a serem seguidas. Primeiramente, será feita uma revisão das funções da programação Web, para que seja definido o que o usuário irá aprender. Posteriormente, será desenvolvida uma plataforma online que ensine cada elemento da programação ao mesmo tempo que o usuário resolva problemas utilizando o que aprendeu. Por fim, serão coletadas as opiniões dos usuários sobre o conteúdo ensinado e sobre a plataforma para que a avaliação acerca da mesma seja realizada.

Ao final deste projeto, espera-se que os usuários adquiram um interesse maior sobre a área de programação Web, tendo a plataforma como uma ideia inicial do que pode ser criado com essa programação. Além disso, como um dos principais resultados, espera-se que eles aprendam, de maneira sucinta, os conceitos ensinados. Este trabalho contribui para a demonstração de como a gamificação pode ser utilizada para o ensino nas escolas como uma forma de atrair a atenção dos alunos para as aulas.